% Marble Diagrams
\newcommand{\op}[1]{
  	\node[draw,rounded corners,thick,fit={(0,.8)(5,1.4)}] (id) {};
  	\node[anchor=base,inner sep=0pt] at (id.center) {#1};
}
\newcommand{\stream}[1]{
	\draw[->,very thick] (0,#1) -- (5,#1);
}
\newcommand{\dotstream}[2]{
	\draw[->,very thick] (0,#1) -- (#2,#1);
	\draw[dotted] (#2,#1) -- (5,#1);
}
\newcommand{\onnext}[4]{
	\draw (#1,#2) node [draw,circle,fill=#3] {\tiny #4};
}
\newcommand{\error}[2]{
	\draw[myrd, line width=1mm] (#1-0.3,#2-0.3) -- (#1+0.3,#2+0.3) (#1+0.3,#2-0.3) -- (#1-0.3,#2+0.3);
}
\newcommand{\complete}[2]{
	\draw[mygr, line width=1mm] (#1,#2-0.2) -- (#1,#2+0.2);
}
\newcommand\mat[2]{
	\node [rectangle,minimum size=5.5mm,inner sep=0pt,outer sep=0pt,thick,fill=mybr,opacity=.8] at (#1,#2) {};
}
\newcommand\llist[2]{
	\node[draw,rounded corners,thick,fit={(#1-0.1,-0.3)(#2+0.1,0.3)}]{};
}

% Dataflow graphs
\tikzset{
	dataflow/.style={draw, fill=myye, rounded corners, thick},  	  	
  	machine/.style={draw, fill=myrd!80, rounded corners, thick, inner sep=.5cm},
  	task/.style={draw, fill=mygr!80, rounded corners, thick, inner sep=.5cm},
  	to/.style={->,>=stealth',shorten >=1pt,semithick},
  	net/.style={to, red, dotted},  	
  	opt/.style={->,>=stealth',shorten >=1pt,semithick,draw=red},
  	point/.style={draw, inner sep=0pt, circle, fill=black},
  	every matrix/.style= 	{ampersand replacement=\&},
  	every node/.style={align=center, anchor=base}
}

% General macros
\newcommand\site[1]{\footnote{\url{#1}}}

\newcommand\myimage[3]{
	\begin{figure}[h!] 
	\centering
 	\includegraphics[scale=#2]{#1}
  	\caption{#3}
	\end{figure}
}

\newcommand\codefig[3]{
	\begin{figure}[h!] 
		\begin{minipage}[c]{0.5\linewidth} 
			\input{diagrams/#1.tikz} 
		\end{minipage}
		\hfill
		\begin{minipage}[c]{0.5\linewidth} 
			#2 
		\end{minipage}		
		\caption{#3}
	\end{figure}
}

\newcommand\mydiag[2]{
	\begin{figure}[h!] 
		\centering
	 	\input{diagrams/#1.tikz}
	  	\caption{#2}
	\end{figure}
}

\newcommand\myop[2]{
	\begin{figure}[h!] 
		\centering
	 	\input{diagrams/operators/#1.tikz}
	  	\caption{#2}
	\end{figure}
}

\newcommand\twofig[3]{
	\begin{figure}[h!]
		\begin{minipage}[c]{0.5\linewidth} \input{diagrams/#1.tikz} \end{minipage}
		\hfill
		\begin{minipage}[c]{0.5\linewidth} \input{diagrams/#2.tikz} \end{minipage}
		\caption{#3}
	\end{figure}
}

\newcommand\myitemNospace[4]{
	\begin{minipage}[c]{0.4\linewidth}
	\item[#1] #2 \\
	\textit{inputs}: #3 \\
	\textit{output}: #4
	\end{minipage}
	\hfill
	\begin{minipage}[c]{0.4\linewidth} \input{diagrams/operators/#1.tikz}\end{minipage}
}
\newcommand\myitem[4]{
	\begin{minipage}[c]{0.4\linewidth}
	\item[#1] #2 \\
	\textit{inputs}: #3 \\
	\textit{output}: #4
	\end{minipage}
	\hfill
	\begin{minipage}[c]{0.4\linewidth} \input{diagrams/operators/#1.tikz} \end{minipage}
	\newline \newline \newline
}

\newcommand\myitemmNospace[4]{
	\begin{minipage}[c]{0.4\linewidth}
	\item[#1] #2 \\
	\textit{inputs}: #3 \\
	\textit{output}: #4
	\end{minipage}
	\hfill
}

\newcommand\myitemm[4]{
	\begin{minipage}[c]{0.4\linewidth}
	\item[#1] #2 \\
	\textit{inputs}: #3 \\
	\textit{output}: #4
	\end{minipage}
	\hfill
	\newline \newline \newline
}

\newcommand\optimization[2]{
	\begin{figure}[h!]

	\begin{minipage}[c]{0.3\linewidth} 		
	\end{minipage}
	\begin{minipage}[c]{0.4\linewidth} 
		\input{diagrams/optimization/#1B.tikz} 
	\end{minipage}
	\begin{minipage}[c]{0.1\linewidth} 
		\includegraphics[scale=0.2]{arrow} 
	\end{minipage}
	\hfill
	\begin{minipage}[c]{0.3\linewidth} 
		\input{diagrams/optimization/#1A.tikz} 
	\end{minipage}
	\caption{#2}
	\end{figure}	
}

% Source Code
\newmintedfile[hs]{hs}{frame=lines,framesep=0.5cm}
\newmintedfile[jv]{java}{frame=lines,framesep=0.5cm}
\newmintedfile[scala]{scala}{frame=lines,framesep=0.5cm}
\newmintedfile[scalas]{scala}{fontsize=\footnotesize,frame=lines,framesep=0.5cm}
\usemintedstyle{friendly}
\newmintedfile[jvs]{java}{fontsize=\footnotesize,frame=lines,framesep=0.5cm}
\usemintedstyle{friendly}
\newmintedfile[jvxs]{java}{fontsize=\tiny,frame=lines,framesep=0.5cm}
\usemintedstyle{friendly}